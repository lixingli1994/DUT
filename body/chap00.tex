% !TEX TS-program = XeLaTeX
% !TEX encoding = UTF-8 Unicode

\chapter*{\hfill 科研学习基本情况 \hfill}
\addcontentsline{toc}{chapter}{科研学习基本情况}
\label{chap00}

\section*{课程学习情况(附成绩单)}
\addcontentsline{toc}{section}{课程学习情况(附成绩单)}

根据硕博连读培养计划,本人已完成硕士阶段培养计划内的所有课程,正在完成博士阶段培养计划内的所有课程。硕士阶段培养计划要求总学分xx分,其中必修课学分xx分,选修课学分x分;博士阶段培养计划要求总学分xx分,其中必修课x分,选修课x分。表1为硕博连读期间目前所选读课程的成绩单:

\begin{table}[htbp]
  \bicaption[tab:xiaoshu]{表}{所修课程和成绩}{Tab.}{Courses completed and grades obtained}
  \centering
  \vspace{0.2cm}
  \zhongwu
  \begin{tabular}{lcccc}
    \toprule
    课程名称 & 学分 & 成绩 & 所属阶段(硕/博士) & 必修/选修 \\
    \midrule
    理论物理研究基础                             & 2            & 95           & 硕士 & 必修\\
    阅读与写作II(全球化研究,西方文学、哲学经典) & 2             & 95         & 硕士  & 必修\\         
    开放量子系统理论 & 2            & P           & 博士& 选修\\
    量子色动力学 & 2            & P           & 博士& 选修\\
    \bottomrule
  \end{tabular}
\end{table}


\section*{参加科研和学术活动情况}
\addcontentsline{toc}{section}{参加科研和学术活动情况}
除了选修上述课程以外,在科研方面,我阅读了大量与自己课题相关的文献并进行了总结。此外,我积极参加各类专业学术会议与讲座进行学习,包括并不限于以下:
\begin{enumerate}
\item 2016东北地区量子物理前言与进展研讨会(辽宁-大连)
\end{enumerate}





