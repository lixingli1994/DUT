% !TEX TS-program = XeLaTeX
% !TEX encoding = UTF-8 Unicode

%%%%%%%%%%%%%%%%%%%%%%%%%%%%%%%%%%%%%%%%%%%%%%%%%%%%%%%%%%%%%%%%%%%%%% 
% 
% 大连理工大学硕士论文 XeLaTeX 模版 —— 字体配置文件 fonts.tex
% 版本:0.82
% 最后更新:2012.05.07
% 修改者:Yuri (E-mail: yuri_1985@163.com)
% 修订者:whufanwei(E-mail: dutfanwei@qq.com) 
% 编译环境1:Ubuntu 12.04 + TeXLive 2011 + Emacs
% 编译环境2:Windows 7 + CTeX v2.9.2.164 + WinEdit
% 
%%%%%%%%%%%%%%%%%%%%%%%%%%%%%%%%%%%%%%%%%%%%%%%%%%%%%%%%%%%%%%%%%%%%%% 

% 英文字体设置特别推荐方案(Windows,需要安装 Adobe 字体),现代
\usepackage{fontspec}
\usepackage{xltxtra,xunicode}
\usepackage[CJKnumber,CJKchecksingle,BoldFont]{xeCJK}
\usepackage{amsmath}
\usepackage{amssymb}
% \usepackage{mathspec}
\setmainfont[Mapping=tex-text]{Times New Roman}
\setsansfont[Mapping=tex-text]{Arial} 
\setmonofont{Consolas}
% \setmathfont{Times New Roman}



% 英文字体设置方案一(Windows,需要安装 LM10 字体),和 LaTeX 默认字体保持一致,经典
% \usepackage{amssymb}
% \usepackage{fontspec}
% \usepackage{amsmath}
% \usepackage[CJKnumber,CJKaddspaces,CJKchecksingle,BoldFont]{xeCJK}
% \usepackage{mathrsfs}   % 一种常用于定义泛函算子的花体字母,只有大写。
% \usepackage{bm}         % 处理数学公式中的黑斜体的宏包
% \setmainfont{LMRoman10-Regular}
% \setsansfont{LMSans10-Regular}
% \setmonofont{LMMono10-Regular}

% 英文字体设置方案二(Linux,使用自带 LM10 字体),和 LaTeX 默认字体保持一致,经典
% \usepackage{fontspec}
% \usepackage{amsmath,amssymb}
% \usepackage[CJKnumber,CJKaddspaces,CJKchecksingle,BoldFont]{xeCJK}
% \usepackage{mathrsfs}   % 一种常用于定义泛函算子的花体字母,只有大写。
% \usepackage{bm}         % 处理数学公式中的黑斜体的宏包
% \setmainfont{LMRoman10}
% \setsansfont{LMSans10}
% \setmonofont{LMMono10}

% 英文字体设置方案三(Linux,使用自带 Nimbus 字体),和 Word 模版字体保持一致,经典
% \usepackage{fontspec}
% \usepackage{mathptmx}
% \usepackage{amsmath,amssymb}
% \usepackage[CJKnumber,CJKaddspaces,CJKchecksingle,BoldFont]{xeCJK}
% \usepackage{mathrsfs}   % 一种常用于定义泛函算子的花体字母,只有大写。
% \usepackage{bm}         % 处理数学公式中的黑斜体的宏包
% \setmainfont{Nimbus Roman No9 L}
% \setsansfont{Nimbus Sans L}
% \setmonofont{Nimbus Mono L}

% 中文字体设置,使用的是 Adobe 字体,保证了在 Adobe Reader / Acrobat 下优秀的显示效果
\setCJKmainfont[BoldFont={Adobe Heiti Std},ItalicFont={Adobe Kaiti Std}]{Adobe Song Std}
\setCJKsansfont{Adobe Heiti Std}
\setCJKmonofont{Adobe Fangsong Std}

% 定义字体名称,可在此添加自定义的字体
\setCJKfamilyfont{song}{Adobe Song Std}
\setCJKfamilyfont{hei}{Adobe Heiti Std}
\setCJKfamilyfont{kai}{Adobe Kaiti Std}
\setCJKfamilyfont{fs}{Adobe Fangsong Std}
\setCJKfamilyfont{xkai}{STXingkai}
\setCJKfamilyfont{hwzsong}{STZhongsong}

% 自动调整中英文之间的空白
% \punctstyle{quanjiao}
\XeTeXlinebreaklocale "zh"      %中文断行
\XeTeXlinebreakskip = 0pt plus 1pt %1pt左右弹性间距
% 其他字体宏包
